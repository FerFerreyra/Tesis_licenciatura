\usepackage[top=1in, left=1.25in, right=1.25in, bottom=1in]{geometry}

%Portada formato tesis UNAM
\usepackage{bachelorstitlepageUNAM}

%%%%%%%%%%%%%%%%%%%%%%%%%%%%%%%%%%%
\usepackage{titlesec} % Quita palabra «Capítulo» de \chapter

\titleformat{\chapter} % Config titlesec
{\Large\bfseries}	% 
{\huge \thechapter}	% 
{20pt}				% 
{\huge}				% 

\usepackage[titletoc]{appendix} % Añade palabra «Apéndice» en Índice y define el ambiente appendices
\usepackage{cite} % Define varios tipos de entrada en archivos *.bib
%%%%%%%%%%%%%%%%%%%%%%%%%%%%%%%%%%%

\usepackage[utf8]{inputenc}
\usepackage[spanish,mexico]{babel}
\usepackage{textgreek}

\usepackage[es-tabla,es-nodecimaldot]{babel}

%Paquetes matemáticas
\usepackage{amsmath}
\usepackage{amssymb}
\usepackage{amsthm}
\usepackage{amsfonts}
\usepackage{graphicx}
\usepackage{mathtools}
\usepackage{amsmath,amssymb,tikz-cd}

\usepackage{multirow}

\usepackage{multicol}



\usepackage{graphicx}

\usepackage{caption}
\captionsetup{font=small} % Tamaño fuente 10pt en caption

%Enumeración
\usepackage{enumerate}
\usepackage{enumitem}   

% Tablas bonitas
\usepackage{booktabs}

% Diagramas
\usepackage{tikz-cd}
\usetikzlibrary{positioning, 
				graphs,
				quotes,
				automata,
				arrows}

% Verbatim
\usepackage{fancyvrb}

%Paquete para escribir código
\usepackage{listings}
\usepackage{xcolor}

%Paquete para referencias
\usepackage{hyperref}
\hypersetup{
	colorlinks=true,
	linkcolor=blue,
	filecolor=magenta,      
	urlcolor=cyan,
	pdftitle={Overleaf Example},
	pdfpagemode=FullScreen,
}

%New colors defined below
\definecolor{codegreen}{rgb}{0,0.6,0}
\definecolor{codegray}{rgb}{0.5,0.5,0.5}
\definecolor{codepurple}{rgb}{0.58,0,0.82}
\definecolor{backcolour}{rgb}{0.95,0.95,0.92}

%Code listing style named "mystyle"
\lstdefinestyle{mystyle}{
	backgroundcolor=\color{backcolour}, commentstyle=\color{codegreen},
	keywordstyle=\color{magenta},
	numberstyle=\tiny\color{codegray},
	stringstyle=\color{codepurple},
	basicstyle=\ttfamily\footnotesize,
	breakatwhitespace=false,         
	breaklines=true,                 
	captionpos=b,                    
	keepspaces=true,                 
	numbers=left,                    
	numbersep=5pt,                  
	showspaces=false,                
	showstringspaces=false,
	showtabs=false,                  
	tabsize=2
}

%"mystyle" code listing set
\lstset{style=mystyle}

\theoremstyle{plain}
\newtheorem{thm}{Teorema}
\newtheorem{prop}{Proposición}
\newtheorem{lem}{Lema}
\newtheorem{cor}{Corolario}
\newtheorem{conj}{Conjetura}

\theoremstyle{definition}
\newtheorem{dfn}{Definición}
\newtheorem{pre}{Pregunta}
\newtheorem{ej}{Ejemplo}

\theoremstyle{remark}
\newtheorem{obs}{Observación}


% Respuesta
\newenvironment{resp}{\renewcommand{\proofname}{Respuesta}\renewcommand{\qedsymbol}{}\begin{proof}}{\end{proof}}

%\usepackage{fixmetodonotes}
% \usepackage{todonotes} % define \todo para agregar cosas por hacer en la tesis. también define \missingfigure que es útil cuando aún no se tiene la gráfica/figura/diagrama que se quiere añadir a la mano.
\usepackage{subfiles}

\newcommand{\py}[1]{{\color{brown}\texttt{#1}}}
\newcommand{\signature}[2]{#2 \xleftarrow{\tt{dom}} #1 \xrightarrow{\tt{cod}} #2}
\renewcommand{\phi}{\varphi}
\renewcommand{\epsilon}{\varepsilon}
\newcommand{\nota}[1]{{\color{red}\textbf{#1}}}
